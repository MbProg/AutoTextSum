The summaries are given to human annotators for evaluation. The annotators are students who attend the same course but are in another work group (?). For evaluation Likert Scales are used. Since refernce summaries don't exist it can't be evaluated by comparing a summary with a gold standard. Furthermore the annotators shouldn't have to read all ... source documents of a summary to judge the summary itself. This process woud be too time-consuming. Instead items are used on the Likert Scale which can be judged by only reading the summary itself. In total there are eleven categories: "Grammaticality", Non-redundancy", Referential clarity", "Focus", "Structure", "Coherence", "Readability", "Information Content", "Spelling", "Length" and "Overall Quality". For each category the annotators should assign a score from 1 (= very poor) to 5 (= very good), a weight and a confidence (both scales also from 1 to 5) of their grading. Each summary is evaluated by four to five different annotators.

Besides the summaries of all groups summaries created by two simple approaches (footnote) are evaluated as well. These summaries serve as baseline summaries. The first approach is ... The second approach is ... 


Most categories seem like any text evaluation categories like "Spelling" and "Grammaticality". Other categories seem especially summary-related. These are the categories "Information Content" and "Focus". They represent the goal of a summary very well which is to present the most important content of the summarized texts. Since all summarized texts in this corpus are about a certain query the focus should be visible, too.

The resulting evaluations can be used for assessing the quality of the summaries produced by our system. It is important for the evaluation that we only work at the nugget extraction. This input is given to another group which then produced the summaries. In this way we are completely responsible for the results in some evaluation categories while other evaluation results also depend on the steps of  building the hierarchy and actually creating a summary. The output which we after the nugget extraction are whole sentences (more about the output in section ...). The summary is then only built out of these sentences. In this way all categories which just operate on a sentence level are completely our responsibility. Among these categories are strictly only the two categories "Spelling" and "Grammaticality". We are also highly responsible for the categories "Information Content", "Focus" and "Non-Redundancy". All extracted sentences should ideally contain importannt information related to the query. Furthermore it can be argued that in the step of nugget extraction nuggets with the same meaning as another nugget are ignored. The categories "Referential Clarity", "Structure" and "Coherence" in comparison are very dependent on the ordering of the sentences. It can be argued that "Referential clarity" is also influenced by the nugget extraction. For sentences with a pronoun the system should also extract the reference sentence. Otherwise the sentence is not well usable in the next steps. This isn't done in the step of nugget extraction, but in later steps. The category "Length" especially depends on the last step, the summary creation. "Readability" and of course "Information Content" are very general categories which can't be assigned to any particular step. The focus of our analysis will be all steps which can be influenced by our work, the nugget extraction. Thus the categories "Structure", "Length" and "Coherence" will only be shortly discussed.

In the following we compare the results of our group with the results of the other groups and the two baseline approaches. Our average overall score is 2.86. The average overall scores of the other groups are 0.39 to 0.74 points better. In contrast to the baseline approaches our summaries are much better. The baseline approaches only have an average overall score of 1.61 and 1.62. So our approach is more than one point better than the baselines. Now we take a closer look at the different categories. "Overall Qualiy" isn't discussec here because it does not highlight a particular aspect of a summary. Compared to the other groups our summaries are worst in all categories except for "Referential Clarity". In the category "Information Content" which is very important for summaries we outdo both baseline approaches significantly at least. The categories we are best at are "Spelling" with ..., "Non-Redundancy" with ... and "Grammaticality" with .... The other groups also perform best at "Grammaticality" and "Spelling"??? This is not surprising since all groups extracted whole sentences for the summarization. These sentences should be mostly grammatical, correctly spelled sentences. Perhaps there are some exceptions since the sentences are taken from forum posts. categories we are worst in are "Structure" with 2.86 points, "Coherence" with 2.88 points and "Information Content" with 2.9 points. "Structure and "Coherence" are also the categories the other groups perforn worst.???
