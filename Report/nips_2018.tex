\documentclass{article}

% if you need to pass options to natbib, use, e.g.:
% \PassOptionsToPackage{numbers, compress}{natbib}
% before loading nips_2018

% ready for submission
\usepackage{nips_2018}

% to compile a preprint version, e.g., for submission to arXiv, add
% add the [preprint] option:
% \usepackage[preprint]{nips_2018}

% to compile a camera-ready version, add the [final] option, e.g.:
% \usepackage[final]{nips_2018}

% to avoid loading the natbib package, add option nonatbib:
% \usepackage[nonatbib]{nips_2018}

\usepackage[utf8]{inputenc} % allow utf-8 input
\usepackage[T1]{fontenc}    % use 8-bit T1 fonts
\usepackage{hyperref}       % hyperlinks
\usepackage{url}            % simple URL typesetting
\usepackage{booktabs}       % professional-quality tables
\usepackage{amsfonts}       % blackboard math symbols
\usepackage{nicefrac}       % compact symbols for 1/2, etc.
\usepackage{microtype}      % microtypography

\title{Automatic text summarization, 2018}

% The \author macro works with any number of authors. There are two
% commands used to separate the names and addresses of multiple
% authors: \And and \AND.
%
% Using \And between authors leaves it to LaTeX to determine where to
% break the lines. Using \AND forces a line break at that point. So,
% if LaTeX puts 3 of 4 authors names on the first line, and the last
% on the second line, try using \AND instead of \And before the third
% author name.

\author{
  David S.~Hippocampus\thanks{Use footnote for providing further
    information about author (webpage, alternative
    address)---\emph{not} for acknowledging funding agencies.} \\
  Department of Computer Science\\
  Cranberry-Lemon University\\
  Pittsburgh, PA 15213 \\
  \texttt{hippo@cs.cranberry-lemon.edu} \\
  %% examples of more authors
  %% \And
  %% Coauthor \\
  %% Affiliation \\
  %% Address \\
  %% \texttt{email} \\
  %% \AND
  %% Coauthor \\
  %% Affiliation \\
  %% Address \\
  %% \texttt{email} \\
  %% \And
  %% Coauthor \\
  %% Affiliation \\
  %% Address \\
  %% \texttt{email} \\
  %% \And
  %% Coauthor \\
  %% Affiliation \\
  %% Address \\
  %% \texttt{email} \\
}

\begin{document}
% \nipsfinalcopy is no longer used

\maketitle

\begin{abstract}
  Today there are many documents, articles, papers and reports available in digital form. These volumes of text are invaluable sources of information and knowledge that need to be effectively summarized to be useful. In automatic text summarization machine learning techniques are often used to generate summaries. A prior step to the generation of summaries is usually the extraction of nuggets. This paper presents the two approaches we use for the extraction of nuggets, as well as a description of their effectiveness and shortcomings. 
  
\end{abstract}

\section{Introduction}
With the dramatic growth of the internet, people are overwhelmed by the tremendous amount of online information and documents. This expansion in availability of data has demanded extensive research in the automatic generation of summaries from a collection of different type of text.\\

\textit{Automatic summarization} is the process of shortening a text document with software, in order to create a summary with the major points of the original document.	     

In general, there  are two different approaches for text summarization:\textit{extraction} and \textit{abstraction}


\begin{center}
  \url{https://cmt.research.microsoft.com/NIPS2018/}
\end{center}

Please read the instructions below carefully and follow them faithfully.

\subsection{Style}


$\sim$$15\%$ more words in the paper compared to earlier years.

Authors are required to use the NIPS \LaTeX{} style files obtainable
at the NIPS website as indicated below. Please make sure you use the
current files and not previous versions. Tweaking the style files may
be grounds for rejection.

\subsection{Retrieval of style files}

The style files for NIPS and other conference information are
available on the World Wide Web at
\begin{center}
  \url{http://www.nips.cc/}
\end{center}
The file \verb+nips_2018.pdf+ contains these instructions and
illustrates the various formatting requirements your NIPS paper must
satisfy.

The only supported style file for NIPS 2018 is \verb+nips_2018.sty+,
rewritten for \LaTeXe{}.  \textbf{Previous style files for \LaTeX{}
  2.09, Microsoft Word, and RTF are no longer supported!}

The \LaTeX{} style file contains three optional arguments: \verb+final+,
which creates a camera-ready copy, \verb+preprint+, which creates a
preprint for submission to, e.g., arXiv, and \verb+nonatbib+, which will
not load the \verb+natbib+ package for you in case of package clash.

\paragraph{New preprint option for 2018}
If you wish to post a preprint of your work online, e.g., on arXiv,
using the NIPS style, please use the \verb+preprint+ option. This will
create a nonanonymized version of your work with the text
``Preprint. Work in progress.''  in the footer. This version may be
distributed as you see fit. Please \textbf{do not} use the
\verb+final+ option, which should \textbf{only} be used for papers
accepted to NIPS.

At submission time, please omit the \verb+final+ and \verb+preprint+
options. This will anonymize your submission and add line numbers to aid
review. Please do \emph{not} refer to these line numbers in your paper
as they will be removed during generation of camera-ready copies.

The file \verb+nips_2018.tex+ may be used as a ``shell'' for writing
your paper. All you have to do is replace the author, title, abstract,
and text of the paper with your own.

The formatting instructions contained in these style files are
summarized in Sections \ref{gen_inst}, \ref{headings}, and
\ref{others} below.

\section{Evaluation}

\subsection{Manual evaluation}

The summaries are given to human annotators for evaluation. The annotators are students who attend the same course but are in another work group (?). For evaluation Likert Scales are used. Since refernce summaries don't exist it can't be evaluated by comparing a summary with a gold standard. Furthermore the annotators shouldn't have to read all ... source documents of a summary to judge the summary itself. This process woud be too time-consuming. Instead items are used on the Likert Scale which can be judged by only reading the summary itself. In total there are eleven categories: "Grammaticality", Non-redundancy", Referential clarity", "Focus", "Structure", "Coherence", "Readability", "Information Content", "Spelling", "Length" and "Overall Quality". For each category the annotators should assign a score from 1 (= very poor) to 5 (= very good), a weight and a confidence (both scales also from 1 to 5) of their grading. Each summary is evaluated by four to five different annotators.

Most categories seem like any text evaluation categories like "Spelling" and "Grammaticality". Other categories seem especially summary-related. These are the categories "Information Content" and "Focus". They represent the goal of a summary very well which is to present the most important content of the summarized texts. Since all summarized texts in this corpus are about a certain query the focus should be visible, too.

The resulting evaluations can be used for assessing the quality of the summaries produced by our system. It is important for the evaluation that we only work at the nugget extraction. This input is given to another group which then produced the summaries. In this way we are completely responsible for the results in some evaluation categories while other evaluation results also depend on the steps of  building the hierarchy and actually creating a summary. The output which we after the nugget extraction are whole sentences (more about the output in section ...). The summary is then only built out of these sentences. In this way all categories which just operate on a sentence level are completely our responsibility. Among these categories are only the two categories "Spelling" and "Grammaticality". Other categories are partially dependent on the choice of sentences. These are the categories "Focus", "Information Content" and "Non-Redundancy". 

%%%%%%%%%%%%%%%%%%%%%%%%%%%%%%%%%%%%%%%% CHECK m min max, treshold..
\section{Implementation}
 \subsection{Second Approach}
For the second approach we used sequence classification to classify each possible candidate as either a nugget or not. More precisely we generated a candidate $s_{i,m}$ for each word $w_i$ with $m$ following words. We iterated the candidate length $m=|s_{i,m}|$ between a minimum length of 5 and a maximum of 20, if the length of the sentence allowed it. So $5<=m<=20$. So in the end we would have every possible candidate in any given sentence. \\
This preprocessing required substantive computational time, so we saved these candidates to the disk to have the whole training set availabe and speed up the training process. We found that only a few percent of possible nugget candidates actually were annotated by at least one worker. This extreme class imbalance can lead to serious issues in the learning process with models models \citep{japkowicz2002class}. To fix this issue we used oversampling of the rare class, i.e. after shuffling all samples we would take 20\% candidates with a label bigger than zero and 80\% random candidates. This makes it much more likely to have at least one positive sample in each batch, so that the model can have a more stable training signal.

We chose this approach because it seems like a intuitive solution to the problem that nuggets can also be incomplete sentences. In many of the target nuggets we were given, the workers annotated only the most important phrases of a sentence. 

To represent the nugget candidates we used sequences of pretrained Word2Vec word embeddings \cite{w2v}, as well as sentence embeddings computed by the Universal Sentence Encoder module in tensorflow \cite{universal2018}.

To identify the correct nuggets we could use classification with two classes: nugget or not a nugget. This makes thet task similar to sentence classification, which is a very well researched problem. The difference being that nuggets can be incomplete sentences. As \citep{zhou2015c} and others have shown, RNNs are among the top performing models in sentence classification together with CNNs, if given enough training data. For the target data we used a treshold of two, so any phrase which was annotated as a nugget by two or more workers would be considered a nugget in this binary format. Also possible would be regression of the number of Mechanical Turk workers annotating each word sequence as a nugget. Which would be equivalent to the original format of our target data.\\

We used GRUs as proposed by \citet{gru2014}, because we have less training data than most state of the art research projects. While the performance of LSTMs and GRUs are very similar, the GRU has a bit fewer parameters to train and is thus less computationally demanding \cite{cnn_rnn_comparative2017}.

\section{General formatting instructions}
\label{gen_inst}

The text must be confined within a rectangle 5.5~inches (33~picas)
wide and 9~inches (54~picas) long. The left margin is 1.5~inch
(9~picas).  Use 10~point type with a vertical spacing (leading) of
11~points.  Times New Roman is the preferred typeface throughout, and
will be selected for you by default.  Paragraphs are separated by
\nicefrac{1}{2}~line space (5.5 points), with no indentation.

The paper title should be 17~point, initial caps/lower case, bold,
centered between two horizontal rules. The top rule should be 4~points
thick and the bottom rule should be 1~point thick. Allow
\nicefrac{1}{4}~inch space above and below the title to rules. All
pages should start at 1~inch (6~picas) from the top of the page.

For the final version, authors' names are set in boldface, and each
name is centered above the corresponding address. The lead author's
name is to be listed first (left-most), and the co-authors' names (if
different address) are set to follow. If there is only one co-author,
list both author and co-author side by side.

Please pay special attention to the instructions in Section \ref{others}
regarding figures, tables, acknowledgments, and references.

\section{Headings: first level}
\label{headings}

All headings should be lower case (except for first word and proper
nouns), flush left, and bold.

First-level headings should be in 12-point type.

\subsection{Headings: second level}

Second-level headings should be in 10-point type.

\subsubsection{Headings: third level}

Third-level headings should be in 10-point type.

\paragraph{Paragraphs}

There is also a \verb+\paragraph+ command available, which sets the
heading in bold, flush left, and inline with the text, with the
heading followed by 1\,em of space.

\section{Citations, figures, tables, references}
\label{others}

These instructions apply to everyone.

\subsection{Citations within the text}

The \verb+natbib+ package will be loaded for you by default.
Citations may be author/year or numeric, as long as you maintain
internal consistency.  As to the format of the references themselves,
any style is acceptable as long as it is used consistently.

The documentation for \verb+natbib+ may be found at
\begin{center}
  \url{http://mirrors.ctan.org/macros/latex/contrib/natbib/natnotes.pdf}
\end{center}
Of note is the command \verb+\citet+, which produces citations
appropriate for use in inline text.  For example,
\begin{verbatim}
   \citet{hasselmo} investigated\dots
\end{verbatim}
produces
\begin{quote}
  Hasselmo, et al.\ (1995) investigated\dots
\end{quote}

If you wish to load the \verb+natbib+ package with options, you may
add the following before loading the \verb+nips_2018+ package:
\begin{verbatim}
   \PassOptionsToPackage{options}{natbib}
\end{verbatim}

If \verb+natbib+ clashes with another package you load, you can add
the optional argument \verb+nonatbib+ when loading the style file:
\begin{verbatim}
   \usepackage[nonatbib]{nips_2018}
\end{verbatim}

As submission is double blind, refer to your own published work in the
third person. That is, use ``In the previous work of Jones et
al.\ [4],'' not ``In our previous work [4].'' If you cite your other
papers that are not widely available (e.g., a journal paper under
review), use anonymous author names in the citation, e.g., an author
of the form ``A.\ Anonymous.''

\subsection{Footnotes}

Footnotes should be used sparingly.  If you do require a footnote,
indicate footnotes with a number\footnote{Sample of the first
  footnote.} in the text. Place the footnotes at the bottom of the
page on which they appear.  Precede the footnote with a horizontal
rule of 2~inches (12~picas).

Note that footnotes are properly typeset \emph{after} punctuation
marks.\footnote{As in this example.}

\subsection{Figures}

\begin{figure}
  \centering
  \fbox{\rule[-.5cm]{0cm}{4cm} \rule[-.5cm]{4cm}{0cm}}
  \caption{Sample figure caption.}
\end{figure}

All artwork must be neat, clean, and legible. Lines should be dark
enough for purposes of reproduction. The figure number and caption
always appear after the figure. Place one line space before the figure
caption and one line space after the figure. The figure caption should
be lower case (except for first word and proper nouns); figures are
numbered consecutively.

You may use color figures.  However, it is best for the figure
captions and the paper body to be legible if the paper is printed in
either black/white or in color.

\subsection{Tables}

All tables must be centered, neat, clean and legible.  The table
number and title always appear before the table.  See
Table~\ref{sample-table}.

Place one line space before the table title, one line space after the
table title, and one line space after the table. The table title must
be lower case (except for first word and proper nouns); tables are
numbered consecutively.

Note that publication-quality tables \emph{do not contain vertical
  rules.} We strongly suggest the use of the \verb+booktabs+ package,
which allows for typesetting high-quality, professional tables:
\begin{center}
  \url{https://www.ctan.org/pkg/booktabs}
\end{center}
This package was used to typeset Table~\ref{sample-table}.

\begin{table}
  \caption{Sample table title}
  \label{sample-table}
  \centering
  \begin{tabular}{lll}
    \toprule
    \multicolumn{2}{c}{Part}                   \\
    \cmidrule(r){1-2}
    Name     & Description     & Size ($\mu$m) \\
    \midrule
    Dendrite & Input terminal  & $\sim$100     \\
    Axon     & Output terminal & $\sim$10      \\
    Soma     & Cell body       & up to $10^6$  \\
    \bottomrule
  \end{tabular}
\end{table}

\section{Final instructions}

Do not change any aspects of the formatting parameters in the style
files.  In particular, do not modify the width or length of the
rectangle the text should fit into, and do not change font sizes
(except perhaps in the \textbf{References} section; see below). Please
note that pages should be numbered.

\section{Preparing PDF files}

Please prepare submission files with paper size ``US Letter,'' and
not, for example, ``A4.''

Fonts were the main cause of problems in the past years. Your PDF file
must only contain Type 1 or Embedded TrueType fonts. Here are a few
instructions to achieve this.

\begin{itemize}

\item You should directly generate PDF files using \verb+pdflatex+.

\item You can check which fonts a PDF files uses.  In Acrobat Reader,
  select the menu Files$>$Document Properties$>$Fonts and select Show
  All Fonts. You can also use the program \verb+pdffonts+ which comes
  with \verb+xpdf+ and is available out-of-the-box on most Linux
  machines.

\item The IEEE has recommendations for generating PDF files whose
  fonts are also acceptable for NIPS. Please see
  \url{http://www.emfield.org/icuwb2010/downloads/IEEE-PDF-SpecV32.pdf}

\item \verb+xfig+ "patterned" shapes are implemented with bitmap
  fonts.  Use "solid" shapes instead.

\item The \verb+\bbold+ package almost always uses bitmap fonts.  You
  should use the equivalent AMS Fonts:
\begin{verbatim}
   \usepackage{amsfonts}
\end{verbatim}
followed by, e.g., \verb+\mathbb{R}+, \verb+\mathbb{N}+, or
\verb+\mathbb{C}+ for $\mathbb{R}$, $\mathbb{N}$ or $\mathbb{C}$.  You
can also use the following workaround for reals, natural and complex:
\begin{verbatim}
   \newcommand{\RR}{I\!\!R} %real numbers
   \newcommand{\Nat}{I\!\!N} %natural numbers
   \newcommand{\CC}{I\!\!\!\!C} %complex numbers
\end{verbatim}
Note that \verb+amsfonts+ is automatically loaded by the
\verb+amssymb+ package.

\end{itemize}

If your file contains type 3 fonts or non embedded TrueType fonts, we
will ask you to fix it.

\subsection{Margins in \LaTeX{}}

Most of the margin problems come from figures positioned by hand using
\verb+\special+ or other commands. We suggest using the command
\verb+\includegraphics+ from the \verb+graphicx+ package. Always
specify the figure width as a multiple of the line width as in the
example below:
\begin{verbatim}
   \usepackage[pdftex]{graphicx} ...
   \includegraphics[width=0.8\linewidth]{myfile.pdf}
\end{verbatim}
See Section 4.4 in the graphics bundle documentation
(\url{http://mirrors.ctan.org/macros/latex/required/graphics/grfguide.pdf})

A number of width problems arise when \LaTeX{} cannot properly
hyphenate a line. Please give LaTeX hyphenation hints using the
\verb+\-+ command when necessary.

\subsubsection*{Acknowledgments}

Use unnumbered third level headings for the acknowledgments. All
acknowledgments go at the end of the paper. Do not include
acknowledgments in the anonymized submission, only in the final paper.

\section*{References}

References follow the acknowledgments. Use unnumbered first-level
heading for the references. Any choice of citation style is acceptable
as long as you are consistent. It is permissible to reduce the font
size to \verb+small+ (9 point) when listing the references. {\bf
  Remember that you can use more than eight pages as long as the
  additional pages contain \emph{only} cited references.}
\medskip

\small

\bibliographystyle{plain}
\bibliography{nips_2018}


\end{document}
